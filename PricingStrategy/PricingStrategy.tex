Based on the the conclusion from chapter \ref{Chapter2} regarding the consumer positioning, and the considerations from the analysis in previous section, we can now determine the pricing strategy.  

\subsection{Criteria for a successful pricing strategy}
The pricing strategy needs to be adapted by the chosen target group; Young and agnostic. Furthermore we need to respect MUJI's philosophy about high quality products and fair prices \cite{FutureGrowth}. Other than that, we might as well select a pricing strategy with respect to the Executive's plan of making prices that only varies with a maximum of 30\% globally \cite{FutureGrowth}. Finally, MUJI needs to differentiate themselves from the competitors. 

\subsection{Perceived value pricing}
My proposal as to which pricing strategy MUJI should use in order to meet the criteria stated above, is the perceived value pricing, more specifically with a high-value strategy. The consumers in the chosen target group are not interested in products with labels and high brand value, they are looking for bargains and quality. One of the biggest global consumer trends for 2016 was that consumers are finding quality in unknown, unadvertised brands, which fits well with MUJI's concept of no brand. This pricing strategy is a good opportunity for MUJI to differentiate themselves from the competitors, offering minimalist quality products at fair prices. Today, the alternative for buying cheap products from IKEA or the like, is to buy overpriced products from high-brand companies. There seem to be a gap in the market covering the need for the consumers who wish to buy products exclusively for the quality, and not for the brand or image. My proposal is consistent with my recommendation in previous chapter illustrated with a positioning map, figure \ref{PositioningMap}. This gap is where MUJI wants to be.  