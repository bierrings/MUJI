The purpose of using Porters 5 Forces is to gain an understanding of the market, including MUJI's position in relation to \textit{new entrants}, \textit{suppliers}, \textit{substitutes}, \textit{customers} and \textit{competitors}.

\subsubsection{Threat of new entrants}
The threat of new entrants in the market for housewares and furniture exists but is not very dominant, because of barriers in terms of economies of scale and high initial investments. The existing players on the market are quite large and have many advantages in relation to cost and production, and potential new entrants would have to invest heavily in order to compete on price and quality.

\subsubsection{Bargaining power of suppliers}
MUJI's bargaining power in relation to the suppliers is high, because of their size and growth. MUJI's revenue was more than 307 million yen in FY2015 with an operating profit of 34,4 million yen, representing an increase of 18,2\% of revenue and 44,4\% of profit \cite[2]{Annual}. This shows that MUJI is growing fast and therefore is becoming an even larger customer for the suppliers.     

\subsubsection{Threat of substitutes}
The threat of substitutes is one of the most important forces to consider, since the consumers in Denmark has become a nation of bargain hunters \cite[35]{ConsumerLifestyles}. An opportunity for MUJI to influence the power of substitutes, is that the danish consumers are getting ready to prioritise quality over price again \cite[35]{ConsumerLifestyles}, which gives rise for an appropriate marketing mix. 

\subsubsection{Bargaining power of customers}
MUJI's bargaining power in relation to the customers is low, because of the many alternatives in the market provided by the competitors. The market has a high elasticity, and therefore it is very important for MUJI to differentiate themselves from their competitors in order to minimise the bargaining power of the customers.

\subsubsection{Competitive rivalry between existing players}
\label{CompetitionPorters}
The competitive rivalry between existing players on the market is high. There are many companies of about the same size and much competition on price. For example in the low-price category where IKEA, Daells Bolighus, Biva etc. are located, there is an intensive competition on price. Likewise there are many high-price companies such as Illums Bolighus, HAY, Designdelicatessen and Ilva competing on brand, design and quality. A third region in the market, consists of smaller companies selling high qualiy products with less focus on brand value. The problem for these companies, is the fact that they do not have advantages in terms of economies of scale, which is why the products are expensive. This is the opportunity for MUJI, because they have advantages in terms of economies of scale and additionally they do not seek high brand value. 

