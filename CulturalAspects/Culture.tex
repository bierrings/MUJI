My recommendation as to which cultural aspects of working in Denmark MUJI should consider and act upon, is presented with a comparison of the difference between Denmark and Japan on various aspects. The differences are also illustrated in appendix \ref{CultureComparison}. 

\subsubsection{Hierarchy and status}
The hierarchy in Japan is very steep compared to Denmark. In Denmark we are actually opponents to steep hierarchies, because we believe that the bureaucracy belongs to the past. In Denmark we strive to be as agile as possible, and to allocate decision rights to the individuals in the various departments. Compared to the culture in Japan we have a very informal culture regarding status
and we do not consider age as a measure of status for example, which is the case in Japan \cite{JapaneseCulture}. 

\subsubsection{Deal or relationship focus}
Relationship before business is a very important aspect of doing business in Japan. In Denmark we are more straightforward and we prioritise to focus on business over relationship.  

\subsubsection{Trust, conflicts and communication}
In Japan trust is something to be gained and no business will be discussed until a relationship between the stakeholders has been established. In Denmark we actually, as default, have trust in strangers until it eventually is devastated. In Japan and Asia, in general, conflicts are to be avoided in order to maintain face. In Denmark we believe it is important to face and solve any potential conflicts. The culture in Japan concerning communication is very indirect, and direct communication is seen as impolite and crude. In Denmark we have a very direct way of communicating, and we consider direct and honest sayings to be respectful and efficient \cite{Communication}.    

\subsubsection{Time and group vs. individual}
In Denmark we are very strict concerning time, and we believe that deadlines and agreements must be respected. In Japan there is a more relaxed attitude to deadlines and agreements. In Denmark the individual takes priority, and in Japan the group takes priority. These aspects are very important to consider because they have great influence on the motivation of the employees. 
