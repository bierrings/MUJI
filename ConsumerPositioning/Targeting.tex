\subsubsection{Analysis and comparison of segments}
In the following section, I will analyse and evaluate the three segments by using the SOCC\footnote{Size, Opportunities, Cost, Competition} analysis. In the analysis I will compare different criteria such as \textit{age}, \textit{size}, \textit{growth}, \textit{proportion and expected future proportion in the overall market}, \textit{type of dwelling} (geographic spread), \textit{income} etc.

\begin{table}[H]
\centering
\caption{Comparison of size of segments in year 2013 - 2014 \cite[15-18]{ConsumerLifestyles}}
\label{SegmentsTable}
\begin{tabular}{l|llll}
\textbf{Segment}        & \textbf{A}   & \textbf{B}     & \textbf{C}    &  \\ \cline{1-5}
Age                         & 18-29   & 30-44     & 45-59     &  \\
Size                        & 847.900 & 1.073.000 & 1.159.000 &  \\
Growth                      & 2,5\%   & -1,5\%    & 0,35\%    &  \\
Proportion in the market    & 15,1\%  & 19,1\%    & 20,6\%    &  \\
Expected proportion in 2020 & 15,7\%  & 17,7\%    & 20,6\%    & 
\end{tabular}
\end{table}

\textbf{Size, growth and proportion of the overall market from year 2013 - 2014.}
As shown in table \ref{SegmentsTable}, the segment with the greatest amount of people is segment C, with a population of 1.159.000 people and a growth rate of 0,35\%. The smallest, but also the fastest growing segment, is segment A with a population of 847.900 people and a growth rate of 2,5\%. The population of segment B was 1.073.000 people and the growth rate was negative with -1,5\%. These trends are expected to continue in the next couple of years, and the difference in the overall proportion for segment A and B are expected to converge even more, which makes segment A quite attractive. A factor that might have a negative effect on the purchasing power of segment A, is that 41\% of people aged 18-25 live with their parents, and only 28\% of people aged 25-29 had their own home in 2014 \cite[12]{ConsumerLifestyles}.
\\\\
\textbf{Residents and income.} As we can see in appendix \ref{ResidentsAge}, segment A represents the largest proportion of people living in apartments. With this I assume that segment A is the segment with the largest proportion of people living in urban areas. Segment B and C represent the largest proportion of people living in detached houses and linked or semi-detached houses, which makes sense, because people in these segments might have children and cars. With this I assume that segment B and C are the segments with the largest proportion of people living in rural areas. These factors have great influence on how the marketing mix should be put together. \par

\forceindent As we can see in table \ref{DisposableIncomeTable}, segment C certainly has the largest proportion of income and the highest average monthly income. This is not necessarily positive because it might indicate that the segment will buy more expensive products than the products MUJI sells. Segment B has the second highest amount of income as well as average, monthly income, but this segment represents people, who on average have more expenses compared to segment A and C, because of children, cars etc. 

\begin{table}[H]
\centering
\caption{Income year 2016 based on appendix \ref{DisposableIncome}}
\label{DisposableIncomeTable}
\begin{tabular}{lllllll}
\multicolumn{1}{l|}{\textbf{Segment}}         & \textbf{A}      & \textbf{B}      & \textbf{C}      &  &  &  \\ \cline{1-4}
\multicolumn{1}{l|}{Amount of income (1.000)} & 100.392.471 kr. & 274.959.609 kr. & 347.929.813 kr. &  &  &  \\
\multicolumn{1}{l|}{Average monthly income}   & 11.573 kr.      & 21.926 kr.      & 24.818 kr.      &  &  &  
\end{tabular}
\end{table}

\textbf{Opportunities of influencing the target group and cost needed in order to confront the segment.} The people in segment A are using Facebook less frequently than before, but Snapchat and Instagram are very popular in this segment \cite[13]{ConsumerLifestyles}. Instagram is especially a good place for branding and inspiration, and can be used to target the segment with more authentic and relatable marketing material, eventually posted by customers \cite{GetReal}. Another obvious place to meet this segment is at the educational institutions. Segment B is highly active on social media with 74\% of consumers aged 35-44 having a presence on social media sites \cite[15]{ConsumerLifestyles}. Facebook and LinkedIn are potential channels to reach this segment, but those channels are not the best channels in terms of inspiring the consumers, because consumers generally use other channels to get inspired. Segment C is less active on social media than the other two segments. This segment focuses more on work and many people in this group work for more than 45 hours a week. This segment, taken the products and philosophy of MUJI into consideration, is more difficult and more expensive to influence compared to segment A and B, who are more present on the channels apposite for inspirational advertising. Social media is one of the best and cheapest ways to confront a target group, and it is optimally to confront the segment with marketing material as authentic and relatable as possible. 
\\\\
\textbf{Competition.} The segment with the least intensive competition is segment A, because I believe there is a gap in the market, missing a low- to mid-priced brand with high quality and minimalist furniture and housewares. This aspect is elaborated in chapter \ref{CompetitionPorters}.


\subsubsection{Choice of target group}
Based on previous analysis, my recommendation for the best possible target group is \textbf{segment A; Young and agnostic}. This is the fastest growing segment, the price- and quality-oriented students and graduates, who are curious, inspiration seeking and agnostic shoppers, who cannot be fooled with labels \cite{Top10Trends}. This segment is not representing the highest amount of income, nor has it the greatest size, but it is definitely the segment where MUJI has the opportunity to provide an excellent marketing mix and establish a competitive advantage. The choice of this target group also supplements the opinion of the Executive of Muji - "We have to get younger customers" - Kei Suzuku \cite{FutureGrowth}.
